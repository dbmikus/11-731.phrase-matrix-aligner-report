% English Reference Grammar
% By Dylan B. Mikus
% Spring 2014: 11-731

\usepackage[utf8]{inputenc}

\usepackage{amsthm, amsmath}
% defining variables
\DeclareMathOperator*{\argmin}{arg\,min}
\DeclareMathOperator*{\argmax}{arg\,max}

% These are the commands that we load in and use elsewhere

%%%%%%%%%%%%%%%%%%%%%%%%%%%%%%%%%%%%%%%%%%%%%%%%%%%%%%%%%%%%%%%
%% These functions are for the grow-diag-final-and algorithm %%
%%%%%%%%%%%%%%%%%%%%%%%%%%%%%%%%%%%%%%%%%%%%%%%%%%%%%%%%%%%%%%%

\newcommand{\wrapSmall}[1]{
  \small
  #1
  \normalsize
}

% This function says that a spot (i,j) is aligned if it has some neighbor
% aligned and if the spot is aligned in either of the alignment vectors.
% As such, it is recursively defined.
\newcommand{\isAlignedFromGrow}{
  \begin{align*}
    g(A^{(k)}, & a^{(k)}, b^{(k)}, i, j) = \\
    & A^{(k)}_{i,j}
    \text{ if } (a_i^{(k)} = j
    \text{ or } b_j^{(k)} = i), \\
    & 0 \text{ otherwise}
  \end{align*}
}

% This function just computes all neighboring points (including diagonals) on a
% square grid.
\newcommand{\neighborsFunc}{
  \begin{align*}
    \texttt{neighbors}(i, j) = &
     \{
           \langle i-1, j   \rangle,
           \ \langle i  , j-1 \rangle,
           \ \langle i+1, j   \rangle,
           \ \langle i  , j+1 \rangle, \\
         & \langle i-1, j-1 \rangle,
           \ \langle i-1, j+1 \rangle,
           \ \langle i+1, j-1 \rangle, \\
         & \ \langle i+1, j+1 \rangle
     \}
  \end{align*}
}

% This function evaluates to true (1) if the given matrix point is in the
% intersection of the two alignment vectors.
% If the point is not in the intersection, then it evaluates to true (1)
% if the matrix point has some neighbor that is aligned, and that point is also
% in the union of the two alignment vectors (i.e., `g` evaluates to true), then
% we return true, otherwise we return false (0).
\newcommand{\isAlignedFromInterOrGrow}{
  \begin{align*}
    f(A^{(k)}, & a^{(k)}, b^{(k)}, i, j) = \\
      1 \text{ if }
         & (a_i^{(k)} = j \text{ and } b_j^{(k)} = i), \\
      \text{else } & \texttt{min}(1,
        \ |\{ \langle i', j' \rangle\ :\ i',j'
            \in \texttt{neighbors}(i,j) \\
         &~~~~ \text{ and } g(A^{(k)}, a^{(k)}, b^{(k)}, i', j') = 1 \}|)
  \end{align*}
}


\newcommand{\growDiagMatrix}{
  \begin{align*}
    A^{(k)}_{i,j} = \text{ case } & f(A^{(k)}, a^{(k)}, b^{(k)}, i, j) \text{ of } \\
        & 1 \Longrightarrow 1 \\
        & 0 \Longrightarrow (1 \text{ if } a^{(k)}_i = j
                               \text{ or } b^{(k)}_j = i)
  \end{align*}
}


%%%%%%%%%%%%%%%%%%%%%%%%%%%%%%%%%%%%%%%%%%%%%%%%%%%%%%%%%%%%%%
%% These functions are for the phrase-based-align algorithm %%
%%%%%%%%%%%%%%%%%%%%%%%%%%%%%%%%%%%%%%%%%%%%%%%%%%%%%%%%%%%%%%

% Creates a sequence with the middle elements collapsed.
\newcommand{\seqSpan}[2]{
  \langle #1 \cdots #2 \rangle
}

% where $c(e)$ counts the number of occurrences of phrase $e$ in training,
% and $c(e,f)$ counts the number of phrase pairs $(e,f)$ in training.
\newcommand{\phrasePairNorm}{
    C(\overline{e},\overline{f})
      = \frac{c(\overline{e},\overline{f})}
              {c(\overline{e})}
}

% This function gets all possible phrases in e, all possible phrases in f, and
% the normalized frequency counts for those phrase pairs.
\newcommand{\allPhrasePairs}{
  \begin{align*}
    P =
      \large\{
        (e_i, & e_j, f_m, f_n,
          C(\seqSpan{e_i}{e_j},\ \seqSpan{f_m}{f_n})) \\
        \ &:\ 0 \leq i \leq j \leq \texttt{len}(e),
           \ 0 \leq m \leq n \leq \texttt{len}(f)
      \large\}
  \end{align*}
}

% This evaluates to the most probable set of phrase pairs that cover the
% entirety of the source sentence f and the target sentence e.
\newcommand{\optPhraseCoverage}{
\begin{align*}
    h(e,f&) = \\
    & Q \subset P \text{ such that } \forall x_1, x_2 \in Q \text{ where } \\
        & \hspace{0.35in} x_1 = (e_i, e_j, f_m, f_n, C(\seqSpan{e_i}{e_j},
                                    \seqSpan{f_m}{f_n})) \\
        & \hspace{0.35in} x_2 = (e'_i, e'_j, f'_m, f'_n, C(\seqSpan{e'_i}{e'_j},
                                        \seqSpan{f'_m}{f'_n})), \\
        & \hspace{0.2in} (\seqSpan{e_i}{e_j}
                                        \cap \seqSpan{e'_i}{e'_j} = \emptyset \\
        & \hspace{0.2in} \text{ and } \seqSpan{f_m}{f_n}
                                        \cap \seqSpan{f'_m}{f'_n} = \emptyset) \\
        & \hspace{0.05in} \text{ and such that }
                             \forall y \in \seqSpan{e_0}{\texttt{len}(e)}, \\
        & \hspace{0.925in}   \forall z \in \seqSpan{f_0}{\texttt{len}(f)} \\
        & \hspace{0.35in} \exists x_3 \in Q, x_3 = (e''_i, e''_j, f''_m, f''_n, \\
        &~~~~~~~~~~~~~~~~~~~~~~~~~~~~~ C(\seqSpan{e''_i}{e''_j},
                                   \seqSpan{f''_m}{f''_n})) \\
        & \hspace{0.45in} \text{ such that } e''_i \leq y \leq e''_j, f''_m \leq z \leq f''_n \\
        \vspace{0.5in}
        & \argmax_{Q \subset P} ( Q \text{ for } \sum_{q \in Q} q.\texttt{prob} )
  \end{align*}
}


\newcommand{\growPhraseMatrix}{
  \begin{align*}
    B_{i,j}^{(k)} = \text{ case } & h(e,f) \text{ of } \\
        & \emptyset \Longrightarrow A_{i,j}^{(k)} \\
        & Q \Longrightarrow 1 \text{ if } \exists x \text{ where }
            x = (\langle e_a, e_b \rangle,
                 \langle f_c, f_d \rangle, p) \\
            & \hspace{.5in} \text{ such that } a \leq i \leq b \text{ and } c \leq j \leq d \\
            & \hspace{.4in} 0 \text{ otherwise}
  \end{align*}
}


\newcommand{\intersectMatrix}{
  \begin{align*}
    C_{i,j}^{(k)} = & 1 \text{ if } (A_{i,j}^{(k)} = 1) and (B_{i,j}^{(k)} = 1) \\
                 & 0 \text{ otherwise}
  \end{align*}
}


% Local Variables:
% mode: latex
% End:
